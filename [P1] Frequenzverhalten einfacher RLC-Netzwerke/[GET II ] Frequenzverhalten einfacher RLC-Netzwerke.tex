%//////////////////////////////////
%/// P R E A M B L E

\documentclass[a4paper, 12pt]{article}

\usepackage[utf8]{inputenc}
\usepackage[singlespacing]{setspace}
\usepackage{amsmath}
\usepackage{mathtools}
\usepackage{caption}
\usepackage{float}
\usepackage{graphicx}
\usepackage{multicol}
\usepackage{gensymb}
\usepackage{breqn}
\usepackage{indentfirst}
\usepackage{tabularx, booktabs}
\newcolumntype{Y}{>{\centering\arraybackslash}X}

\usepackage{pdfpages}

\usepackage{multicol}
\usepackage{supertabular}

\usepackage{svg}

\widowpenalty = 4500
\clubpenalty  = 4500

\setlength{\jot}{10pt} %indents

\usepackage{circuitikz}

%%
%% Path settings
%%
\graphicspath{ {./graphics/} }


%//////////////////////////////////
%/// D O C U M E N T
\begin{document}

%%%%%%%%%%%%%%%%%%%%%%%%%%%%%%%%%%%%%
  %\begin{titlepage}
    %\clearpage
    %\maketitle
    %\thispagestyle{empty}
  %\end{titlepage}
  \includepdf{./titlepage/titlepage.pdf}
%%%%%%%%%%%%%%%%%%%%%%%%%%%%%%%%%%%%%

\section{Vorbereitungsaufgaben}
\subsection{}

  \begin{center}
    \begin{align*}
      i(t) = \hat{I} \cdot \cos(\omega t + \phi_i)
    \end{align*}
  \end{center}

  \vspace{0.013155617496424828\columnwidth}


    %Resistor
  (3)\\
  \begin{center}
    \begin{circuitikz}[european voltages, european resistors]
      \draw (0,0) to[R, l=$R$, i=$i(t)$, v=$u_R$, color = black] (2,0);
    \end{circuitikz}

    \begin{align*}
      u_R(t)  & = R \cdot i(t)\\
              & = \underbrace{R \cdot \hat{I}}_{ \mathclap{\hat{U}_R }} \cdot \cos(\omega t + \phi_i)\\
      \intertext{Da sich die Phase nicht ändert, gilt außerdem $\phi_i = \phi_u$ und somit:}
      \Aboxed{u_R(t)  & = \hat{U} \cdot \cos(\omega t + \phi_u)}\\
    \end{align*}

  \end{center}

  (4)\\
  \begin{center}
    %inductance
    \begin{circuitikz}[european voltages, european inductors]
      \draw (0,0) to[L, l=$L$, i=$i(t)$, v=$u_L$, color = black] (2,0);
    \end{circuitikz}

    \begin{align*}
      u_L(t)  & = L \cdot \frac{\text{d}i(t)}{\text{d} t}\\
              & = L  \cdot \hat{I} \cdot \frac{\text{d}}{\text{d} t}\left( \cos(\omega t + \phi_i)\right)\\
              & = -\underbrace{ \omega \cdot L \cdot \hat{I} }_{ \mathclap{\hat{U}_L} } \cdot \sin(\omega t + \phi_i)\\
      \intertext{Um die Spannung ($-\sin{x}$) wieder durch $\cos{x}$ auszudrücken, muss auf den ursprünglichen Phasenwinkel $\pi / 2$ addiert werden:}
      \Aboxed{ u_L(t)  & = \hat{U}_L \cdot \cos(\omega t + \underbrace{\phi_i + \frac{\pi}{2}}_{ \mathclap{\phi_u}})}\\
    \end{align*}

  \end{center}

  (5)\\
  \begin{center}
    %capacitance
    \begin{circuitikz}[european voltages, european resistors]
      \draw (0,0) to[C, l=$C$, i=$i(t)$, v=$u_C$, color = black] (2,0);
    \end{circuitikz}

    \begin{align*}
      u_C(t) & = \frac{\hat{I}}{C} \cdot \int_0^t{i(t) \text{d}t}\\
             & = \frac{\hat{I}}{C} \cdot \int_0^t{ \cos{(\omega t + \phi_i)} \text{d}t}\\
             & = \frac{\hat{I}}{C} \cdot \frac{1}{\omega} [\sin{(\omega t + \phi_i)}]_0^t + \underbrace{U_0}_{\mathclap{\text{initialer Ladezustand}}}\\
             & = \underbrace{\frac{\hat{I}}{C \cdot \omega}}_{\mathclap{\hat{U}_C}} [\sin{(\omega t + \phi_i)} - \sin{(\phi_i)}] + U_0\\
       \intertext{Um die Spannung ($\sin{x}$) wieder durch $\cos{x}$ auszudrücken, muss von dem ursprünglichen Phasenwinkel $\pi / 2$ subtrahiert werden:}
     \Aboxed{ u_C(t)  & = \hat{U}_C \cdot \left(\cos(\omega t + \underbrace{\phi_i - \frac{\pi}{2}}_{\mathclap{\phi_u}}) - \cos{(\underbrace{\phi_i - \frac{\pi}{2}}_{\mathclap{\phi_u}})}\right) + U_0}\\
    \end{align*}

  \end{center}

%1.2
\subsection{}
  (3)\\
  \begin{gather*}
    u_R(t)   = R \cdot i(t)\\
    i(t)      = \frac{u(t)}{R}\\
  \end{gather*}

  (4)\\
  \begin{gather*}
    u_L(t)  = L \cdot \frac{\text{d}i(t)}{\text{d} t}\\
    d i(t)  = \frac{1}{L} \cdot u_L(t) dt\\

    \begin{align*}
      \begin{center} % broken here :)
      i(t)  & = \frac{1}{L} \cdot \int{u_L(t) dt}\\
      i  & = \frac{1}{L} \cdot \int{u_L(t) dt}\\
    \end{align*}
    \end{center}
  \end{gather*}
  (5)\\

%1.3
\subsection{}
\subsection{}
\subsection{}
\subsection{}
\subsection{}
\end{document}
