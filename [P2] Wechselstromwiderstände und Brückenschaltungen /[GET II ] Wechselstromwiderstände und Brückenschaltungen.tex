%//////////////////////////////////
%/// P R E A M B L E

\documentclass[a4paper, 12pt]{article}

\usepackage[utf8]{inputenc}
\usepackage[singlespacing]{setspace}
\usepackage{amsmath}
\usepackage{mathtools}
\usepackage{caption}
\usepackage{float}
\usepackage{graphicx}
\usepackage{multicol}
\usepackage{gensymb}
\usepackage{breqn}
\usepackage{indentfirst}
\usepackage{siunitx}
\usepackage{tabularx, booktabs}
\newcolumntype{Y}{>{\centering\arraybackslash}X}

\usepackage{pdfpages}

\usepackage{multicol}
\usepackage{supertabular}

\usepackage{svg}

\widowpenalty = 4500
\clubpenalty  = 4500

\setlength{\jot}{10pt} %indents


\newcommand*\dif{\mathop{}\!\mathrm{d}}
%\newcommand{\euler}{\mathrm{e}}
%\newcommand{\ramuno}{\mathrm{j}}

%%========================================
%% circuitikz properties
\usepackage[european, straightvoltages]{circuitikz}
%\ctikzvalof{voltage/distance from node = .2}
%\ctikzset{voltage/distance from node  =.5}% in \pgf@circ@Rlen units
%\ctikzset{voltage/distance from line  =.25}% pos. between 0 and 1
%\ctikzset{voltage/bump b/.initial     =1.5}%

\ctikzset{current/distance            = .618}


%%========================================

%%
%% Path settings
%%
\graphicspath{ {./graphics/} }


%//////////////////////////////////
%/// D O C U M E N T
\begin{document}

%%%%%%%%%%%%%%%%%%%%%%%%%%%%%%%%%%%%%
  %\includepdf{Deckblatt.pdf}
  \includepdf{./titlepage/titlepage.pdf}
%%%%%%%%%%%%%%%%%%%%%%%%%%%%%%%%%%%%%

\section{Vorbereitungsaufgaben}

  \subsection{}
    \begin{center}
      \includegraphics[scale=0.5]{./R/2_1/RLC_Zeiger_Impedanz.pdf}\\
      \includegraphics[scale=0.5]{./R/2_1/RLC_Zeiger_Admittanz.pdf}
    \end{center}

  \subsection{}

    \begin{center}
      \begin{circuitikz}

        \draw (0,0) to[R, l=$R$, o-] (2,0);
        \draw (2,0) to[L, l=$L$, i=$\underline{I}$] (6,0);
        \draw (6,0) to[C, l=$C$, -o] (8,0);

      \end{circuitikz}
    \end{center}

    \begin{gather*}
      \underline{I} = \frac{\underline{U}}{\underline{Z}},\,\ \underline{U} = \hat{U} \cdot e^{j(\omega t + \phi_u)}\\
      \underline{Z} = R + j \omega L + \frac{1}{j \omega C}\\
      \underline{I} = \hat{U} \cdot \frac{e^{j(\omega t + \phi_u)}}{R+j \omega L + \frac{1}{j \omega C}}
    \end{gather*}



\section{Versuchsaufgaben}

\end{document}
