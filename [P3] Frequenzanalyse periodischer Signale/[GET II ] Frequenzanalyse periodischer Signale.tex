%//////////////////////////////////
%/// P R E A M B L E

\documentclass[a4paper, 12pt]{article}

\usepackage[utf8]{inputenc}
\usepackage[singlespacing]{setspace}
\usepackage{amsmath}
\usepackage{mathtools}
\usepackage{caption}
\usepackage{float}
\usepackage{booktabs}
\usepackage{graphicx}
\usepackage{multicol}
\usepackage{gensymb}
\usepackage{breqn}
\usepackage{indentfirst}
\usepackage{siunitx}
\usepackage{tabularx, booktabs}
\newcolumntype{Y}{>{\centering\arraybackslash}X}

\usepackage{pdfpages}

\usepackage{multicol}
\usepackage{supertabular}

\usepackage{svg}

\widowpenalty = 4500
\clubpenalty  = 4500

\setlength{\jot}{10pt} %indents


\newcommand*\dif{\mathop{}\!\mathrm{d}}
\newcommand*\shortminus{\scalebox{0.5}[1.0]{\( - \)}}
%\newcommand{\euler}{\mathrm{e}}
%\newcommand{\ramuno}{\mathrm{j}}

%%========================================
%% circuitikz properties
\usepackage[european, straightvoltages]{circuitikz}
%\ctikzvalof{voltage/distance from node = .2}
%\ctikzset{voltage/distance from node  =.5}% in \pgf@circ@Rlen units
%\ctikzset{voltage/distance from line  =.25}% pos. between 0 and 1
%\ctikzset{voltage/bump b/.initial     =1.5}%

\ctikzset{current/distance            = .618}


%%========================================

%%
%% Path settings
%%
\graphicspath{ {./graphics/} }


%//////////////////////////////////
%/// D O C U M E N T
\begin{document}

%%%%%%%%%%%%%%%%%%%%%%%%%%%%%%%%%%%%%
  %\includepdf{Deckblatt.pdf}
  \includepdf{./titlepage/titlepage.pdf}
%%%%%%%%%%%%%%%%%%%%%%%%%%%%%%%%%%%%%

\section{Vorbereitungsaufgaben}

  % 2.1
  \subsection{}
    \begin{center}
      \includegraphics[scale=0.5]{./R/2_1/2_1_function.pdf}
    \end{center}

    \begin{align*}
      \underline{X}_\nu &= \frac{1}{T_1} \cdot \int_{T_1}{x(t) \cdot e^{-(j\nu\cdot\omega_1t)}\dif t}\\
      &= \frac{1}{T_1} \cdot \int_{\shortminus\frac{\tau}{2}}^{\frac{\tau}{2}}{ X_m \cdot e^{-(j\nu\cdot\omega_1t)}\dif t}\\
      &= - \frac{X_m}{T_1 \cdot j \nu \omega_1 } \cdot \left[  e^{-(j\nu\cdot\omega_1t)} \right]^{\frac{\tau}{2}}_{\shortminus\frac{\tau}{2}}\\
      &= -\frac{X_m}{T_1 \cdot j \nu \omega_1 } \cdot \left ( e^{-j\nu\cdot\omega_1 \frac{\tau}{2}} - e^{j\nu\cdot\omega_1 \frac{\tau}{2}} \right )
      %
      \intertext{$\omega_1 = \frac{2 \pi}{T_1}$ und Erweiterung mit $\frac{-1}{-1}$:}
      \underline{X}_\nu &= \frac{X_m}{2 j \pi \nu } \cdot \left ( e^{j\nu\cdot \pi \frac{\tau}{T_1}} - e^{-j\nu\cdot \pi \frac{\tau}{T_1}} \right )\\
      &= \frac{X_m}{\pi \nu } \cdot \frac{\left ( e^{j\nu\cdot \pi \frac{\tau}{T_1}} - e^{-j\nu\cdot \pi \frac{\tau}{T_1}} \right )}{2 j}
    \end{align*}

    \begin{gather*}
      \text{\small{ mit $ \frac{\left ( e^{jx} - e^{-jx} \right )}{2 j} = \sin{(x)}$ und $\frac{\tau}{T_1} = D$: }}\notag\\
      \underline{X}_\nu = \frac{X_m}{\pi \nu } \cdot \sin{(\pi \nu D)}
    \end{gather*}

    \begin{gather*}
        \text{\small Erweitert man wieder mit $\frac{D}{D}$ erhält man das Bild einer Spaltfunktion $\text{si}(x)=\frac{\sin x }{x}$: }\notag\\
        \underline{X}_\nu = D \cdot X_m \cdot \frac{\sin(\pi \nu D)}{\pi \nu D} = D \cdot X_m \cdot \text{si}(\pi \nu D)
      %\shortintertext{or}
    \end{gather*}

    \begin{gather*}
        \text{\small Als reele Reihe:}\notag\\
        x(t) = X_0 + \sum^{\infty}_{\nu=1}{\hat{X}_\nu \cos(\nu \cdot \omega_1 t + \phi_\nu )}\\
        X_0 = \frac{1}{T_1} \cdot \int_{T_1}{x(t) \dif t} = \frac{X_m}{2}\\
    \end{gather*}

    \begin{gather*}
      \text{\small Aus der komplexen Reihendarstellung folgt}\notag\\
      b_\nu = -2 \cdot \text{Im}(\underline{X}_\nu) = 0\\
      \hat{X}_\nu = \sqrt{a^2_\nu + b^2_\nu} = 2 \cdot \mid \underline{X}_\nu\mid \implies a_\nu = 2 \cdot \mid D \cdot X_m \cdot \text{si}(\nu \pi D)\mid \\
      \text{\small $\phi_\nu$ hängt nur vom Wert von $\text{si}{(\nu \pi D)}$ ab, da $\underline{X}_\nu$ rein reell ist:}\notag\\
      \phi_\nu =
      \begin{dcases}
      0 & ; \nu = \frac{4k+1}{2D} \\
      \pi & ; \nu =\frac{4k-1}{2D} \\
      \text{n.d.} & ; \text{sonst}
      \end{dcases}
      %\text{\small also}\notag\\
      %a_\nu = \hat{X}_\nu, \,\
      %\phi_\nu = 0
    \end{gather*}

    \begin{gather*}
      \text{\small Somit ist}\notag\\
      x(t) = \frac{X_m}{2} + \sum^{\infty}_{\nu=1}{2 D X_m \cdot \mid \text{si}(\pi \nu D) \mid \cdot \cos(\nu \cdot \frac{2\pi}{T_1} \cdot t + \phi_\nu)}
    \end{gather*}

    \vspace{0.021276873\paperheight}

    \begin{center}
      \includegraphics[scale=0.5]{./R/2_1/2_1_Reihe.pdf}
    \end{center}

    \begin{gather*}
      \text{\small Effektivwert:}\notag\\
      X_{\text{eff}} = \sqrt{ \frac{1}{T_1} \cdot \int_{T_1}{x^2(t) \dif t} } = \sqrt{ \frac{X_m^2}{T_1} \cdot \int_{\shortminus\frac{\tau}{2}}^{\frac{\tau}{2}}{1 \dif t} }\\
      X_{\text{eff}}= X_m \cdot \sqrt{\frac{\tau}{T_1}} = X_m \cdot \sqrt{D}
    \end{gather*}

    %\vspace{0.021276873\paperheight}

    \begin{center}
      \bgroup
      \def\arraystretch{1.6180339887498948}
        \begin{tabular}{@{}cccc@{}}
        \toprule
        \multicolumn{3}{c}{$D = \frac{1}{2}, X_{\text{eff}}=\frac{X_m}{\sqrt{2}}$} \\ \midrule
        $\nu$      & $\underline{X}_\nu$   & $\hat{X}_\nu$ & $\phi_\nu$ \\ \hline
        1          &  $\frac{1}{\pi} X_m$      &      $\frac{2}{\pi} X_m$&        $0$\\
        2          &  $-$         &       $-$      & $-$  \\
        3          &  $\shortminus\frac{1}{3\pi} X_m$         &   $\frac{2}{3\pi} X_m$   &        $\pi$\\
        4          &  $-$         &       $-$    &$-$    \\
        5          &  $\frac{1}{5\pi} X_m$      &       $\frac{2}{5\pi} X_m$    &  $0$ \\
        6          &  $-$         &       $-$    & $-$   \\
        7          &  $\shortminus\frac{1}{7\pi} X_m$         &          $\frac{2}{7\pi}X_m$  & $\pi$\\
        8          &  $-$         &       $-$     & $-$   \\
        9          &  $\frac{1}{9\pi} X_m$         &       $\frac{2}{9\pi} X_m$   &  $0$   \\
        10         &  $-$         &       $-$       & $-$ \\
        11         &  $\shortminus\frac{1}{11\pi} X_m$         &       $\frac{2}{11\pi} X_m$  &   $\pi$ \\
        12         &  $-$         &       $-$      & $-$  \\
        13         &  $\frac{1}{13\pi} X_m$         &      $\frac{2}{13\pi} X_m$ &   $0$   \\
        14         &  $-$         &      $-$       & $-$  \\
        15         &  $\shortminus\frac{1}{15\pi} X_m$         &      $\frac{2}{15\pi} X_m$   &  $\pi$  \\
        16         &  $-$         &       $-$      & $-$  \\ \bottomrule
        \end{tabular}
        \hspace{0.6180339887498948cm}
        \begin{tabular}{@{}ccc@{}}
        \toprule
        \multicolumn{3}{c}{$D = \frac{1}{4}, X_{\text{eff}}=\frac{X_m}{2}$} \\ \midrule
        $\nu$      & $\hat{X}_\nu$   & $\phi_\nu$ \\ \hline
        1          &  $\frac{\sqrt{2}}{\pi} X_m$      &      $0$        \\
        2          &  $\frac{1}{\pi} X_m$         &       $0$        \\
        3          &  $\frac{\sqrt{2}}{3\pi} X_m$         &   $0$           \\
        4          &  $-$         &       $-$        \\
        5          &  $\frac{\sqrt{2}}{5\pi} X_m$      &       $\pi$      \\
        6          &  $\frac{1}{3\pi} X_m$         &       $\pi$      \\
        7          &  $\frac{\sqrt{2}}{7\pi} X_m$         &          $\pi$   \\
        8          &  $-$         &       $-$       \\
        9          &  $\frac{\sqrt{2}}{9\pi} X_m$         &       $0$      \\
        10         &  $\frac{1}{5\pi} X_m$         &       $0$       \\
        11         &  $\frac{\sqrt{2}}{11\pi} X_m$         &       $0$      \\
        12         &  $-$         &       $-$       \\
        13         &  $\frac{\sqrt{2}}{13 \pi} X_m$         &      $\pi$       \\
        14         &  $\frac{1}{7\pi} X_m$         &      $\pi$        \\
        15         &  $\frac{\sqrt{2}}{15\pi} X_m$         &      $\pi$       \\
        16         &  $-$         &       $-$       \\ \bottomrule
        \end{tabular}
        \hspace{0.6180339887498948cm}
        \begin{tabular}{@{}ccc@{}}
        \toprule
        \multicolumn{3}{c}{$D = \frac{1}{8}, X_{\text{eff}}=\frac{X_m}{\sqrt{8}}$} \\ \midrule
        $\nu$      & $\hat{X}_\nu$   & $\phi_\nu$ \\ \hline
        1          &  $\frac{\sqrt{2-\sqrt{2}}}{\pi} X_m$      &      $0$        \\
        2          &  $\frac{\sqrt{2}}{2\pi}X_m$         &       $0$        \\
        3          &  $\frac{\sqrt{2 + \sqrt{2}}}{3\pi}X_m$         &   $0$           \\
        4          &  $\frac{1}{2}X_m$         &       $0$        \\
        5          &  $\frac{\sqrt{2 + \sqrt{2}}}{5\pi}X_m$      &       $0$      \\
        6          &  $\frac{\sqrt{2}}{16\pi}X_m$         &       $0$      \\
        7          &  $\frac{\sqrt{2 - \sqrt{2}}}{7\pi}X_m$         &          $0$   \\
        8          &  $-$         &       $-$       \\
        9          &  $\frac{\sqrt{2 - \sqrt{2}}}{9\pi}X_m$         &       $\pi$      \\
        10         &  $\frac{\sqrt{2}}{5\pi}X_m$         &       $\pi$       \\
        11         &  $\frac{\sqrt{2 + \sqrt{2}}}{11\pi}X_m$         &       $\pi$      \\
        12         &  $\frac{1}{6\pi}X_m$         &       $\pi$       \\
        13         &  $\frac{\sqrt{2 + \sqrt{2}}}{13\pi}X_m$         &      $\pi$       \\
        14         &  $\frac{\sqrt{2}}{7\pi}X_m$         &      $\pi$        \\
        15         &  $\frac{\sqrt{2 - \sqrt{2}}}{15\pi}X_m$         &      $\pi$       \\
        16         &  $-$         &       $-$       \\ \bottomrule
        \end{tabular}
        \egroup
      \end{center}

  % 2.2
  \subsection{}
    \begin{center}
      \includegraphics[scale=0.5]{./R/2_2/2_2_function.pdf}
    \end{center}

    \begin{align*}
      \underline{X}_\nu &= \frac{1}{T_1} \cdot \int_{\shortminus\frac{T_1}{2}}^{\frac{T_1}{2}}{ X_m \cos{(\omega_0 t) \cdot e^{-(j\nu\omega_0 t)}} \dif t }
      &= \frac{X_m}{T_1} \cdot \left [ \frac{e^{-(j\nu\omega_0 t)}}{1} \right]_{\shortminus\frac{T_1}{2}}^{\frac{T_1}{2}}
    \end{align*}


  % 2.3
  \subsection{}
    \begin{center}
      \begin{circuitikz}

        \draw (0,0) to[V, v=$u_0(t)$] (0,-3);
        \draw (0,0) to[R, l=$R_1$, ] (5,0);
        %\draw (2,0) to[L, l=$L$, i=$\underline{I}$] (6,0);
        \draw (5,0) to[C, l=$C$, *-*] (5,-3);
        \draw (7,0) to[C, l=$R_2$, *-*] (7,-3);
        \draw (0,-3) to[short] (9.25,-3);
        \draw (5,0) to[short] (9.25,0);
        \draw (9.25,0) to[open, v^=$u_2(t)$, o-o] (9.25,-3);

      \end{circuitikz}
    \end{center}

    \begin{gather*}
      \underline{U}_2 = \underline{U_0} \cdot \dfrac{ \frac{1}{ j \omega C + \frac{1}{R_2} }  }{ R_1 + \frac{1}{ j \omega C + \frac{1}{R_2} } } = \frac{\underline{U}_0 \cdot R_2}{R_1+R_2+ j \omega C R_1 R_2}\\
      \text{\small Betrag:}\\
      \hat{U}_2 = \frac{\hat{U}_0 \cdot R_2}{\sqrt{(R_1+R_2)^2 + (\omega C R_1 R_2)^2}}\\\\
      \text{\small Phase:}\\
      \phi_{\underline{U}_2} = \phi_{\underline{U}_0} - \arctan{ \frac{\omega C R_1 R_2}{R_1+R_2} }
    \end{gather*}

\section{Versuchsaufgaben}
  % 3.1
  \subsection{}



  % 3.2
  \subsection{}

  % 3.3
  \subsection{}


\end{document}
